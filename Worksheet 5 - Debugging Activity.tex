%%%%%%%%%%%%%%%%%%%%%%%%%%%%%%%%%%%%%%%%%
% Short Sectioned Assignment
% LaTeX Template
% Version 1.0 (5/5/12)
%
% This template has been downloaded from:
% http://www.LaTeXTemplates.com
%
% Original author:
% Frits Wenneker (http://www.howtotex.com)
%
% License:
% CC BY-NC-SA 3.0 (http://creativecommons.org/licenses/by-nc-sa/3.0/)
%
%%%%%%%%%%%%%%%%%%%%%%%%%%%%%%%%%%%%%%%%%

%----------------------------------------------------------------------------------------
%	PACKAGES AND OTHER DOCUMENT CONFIGURATIONS
%----------------------------------------------------------------------------------------

\documentclass[paper=a4, fontsize=11pt]{scrartcl} % A4 paper and 11pt font size

\usepackage[T1]{fontenc} % Use 8-bit encoding that has 256 glyphs
\usepackage{fourier} % Use the Adobe Utopia font for the document - comment this line to return to the LaTeX default
\usepackage[english]{babel} % English language/hyphenation
\usepackage{amsmath,amsfonts,amsthm} % Math packages
\usepackage{titling}
\usepackage{lipsum} % Used for inserting dummy 'Lorem ipsum' text into the template

\usepackage{multicol}
\setlength{\columnsep}{1cm}

\usepackage{enumitem}

\usepackage{listings}
\usepackage{framed}
\usepackage{setspace} %\begin{doublespacing} \end...

\usepackage{sectsty} % Allows customizing section commands
\allsectionsfont{\centering \normalfont\scshape} % Make all sections centered, the default font and small caps

\usepackage{fancyhdr} % Custom headers and footers
\pagestyle{fancyplain} % Makes all pages in the document conform to the custom headers and footers
\fancyhead{} % No page header - if you want one, create it in the same way as the footers below
\fancyfoot[L]{} % Empty left footer
\fancyfoot[C]{} % Empty center footer
\fancyfoot[R]{\thepage} % Page numbering for right footer
\renewcommand{\headrulewidth}{0pt} % Remove header underlines
\renewcommand{\footrulewidth}{0pt} % Remove footer underlines
\setlength{\headheight}{13.6pt} % Customize the height of the header

\numberwithin{equation}{section} % Number equations within sections (i.e. 1.1, 1.2, 2.1, 2.2 instead of 1, 2, 3, 4)
\numberwithin{figure}{section} % Number figures within sections (i.e. 1.1, 1.2, 2.1, 2.2 instead of 1, 2, 3, 4)
\numberwithin{table}{section} % Number tables within sections (i.e. 1.1, 1.2, 2.1, 2.2 instead of 1, 2, 3, 4)

\setlength\parindent{0pt} % Removes all indentation from paragraphs - comment this line for an assignment with lots of text

%----------------------------------------------------------------------------------------
%	TITLE SECTION
%----------------------------------------------------------------------------------------

\newcommand{\horrule}[1]{\rule{\linewidth}{#1}} % Create horizontal rule command with 1 argument of height

\title{	
\normalfont \normalsize 
\textsc{Northridge Preparatory School} \\ [25pt] % Your university, school and/or department name(s)
\horrule{0.5pt} \\[0.4cm] % Thin top horizontal rule
\huge Worksheet\#5 : Debugging Exercise \\ % The assignment title
\horrule{2pt} \\[0.5cm] % Thick bottom horizontal rule
}

\preauthor{}
\postauthor{}
\author{} % Your name

\predate{}
\postdate{}
\date{} % Today's date or a custom date

\begin{document}

\maketitle % Print the title

%----------------------------------------------------------------------------------------
%	PROBLEM 1
%----------------------------------------------------------------------------------------

\section{Coding Without Comments -- Just Don't!}

     Get in the habit of adding comments when you write code.  Not only will it help others who read your code to understand what it is supposed to do, but it will also help YOU to remember this when you revisit your programs months after you've written them.
     
     
\subsection{No Comments...?}

Read the following faulty code, and try to figure out what this program is trying to do.


\begin{framed}
\begin{lstlisting}
def newfib(x,y):

    a = x + y

    b = x + b

    c = z + b

    d = b * c

    e = c + d

    a,b,x,d,e
\end{lstlisting}
\end{framed}

\begin{enumerate}

\item What does this program do? \rule{9cm}{0.15mm}\\

\rule{13.5cm}{0.15mm}



\pagebreak

\item There are 5 errors in this code.  Try to find them all.  Here it is again:
\begin{framed}
\begin{lstlisting}
def newfib(x,y):

    a = x + y

    b = x + b

    c = z + b

    d = b * c

    e = c + d

    a,b,x,d,e
\end{lstlisting}
\end{framed}

\begin{enumerate}

\item Error 1:\rule{11cm}{0.15mm}\\

\item Error 2:\rule{11cm}{0.15mm}\\

\item Error 3:\rule{11cm}{0.15mm}\\

\item Error 4:\rule{11cm}{0.15mm}\\

\item Error 5:\rule{11cm}{0.15mm}\\

\end{enumerate}

\end{enumerate}

This exercise is intended to be hard, if not impossible.  It's not clear what the program is supposed to do (though the name and general structure gives a small clue...), so you can't really tell what's wrong unless you get an explicit error message.  A few comments would have really helped!


\pagebreak

\subsection{Comments!}

     Try is again with another block of code, this time with comments:
     
\begin{framed}
\begin{lstlisting}
#Given initial velocity & time, return displacement of freefall

def freef(v_0,t):

    a = -9.0     #gravity acceleration in m/s

    d = v0*t + 0.5g(t^2)  #basic displacement formula

    d
\end{lstlisting}
\end{framed}

\begin{enumerate}

\item What does this program do? \rule{9cm}{0.15mm}\\

\rule{13.5cm}{0.15mm}\\

\item There are 5 errors in this code.  Can you find them all?\\

\begin{enumerate}

\item Error 1:\rule{11cm}{0.15mm}\\

\item Error 2:\rule{11cm}{0.15mm}\\

\item Error 3:\rule{11cm}{0.15mm}\\

\item Error 4:\rule{11cm}{0.15mm}\\

\item Error 5:\rule{11cm}{0.15mm}\\

\end{enumerate}

Technically, you shouldn't clutter up your code with too many comments, since that also makes it hard to read.  But how many comments are too many?  That's something you'll eventually learn through practice.  In the meantime, I suggest erring on the side of over-commenting.  Pound for pound, it will take you as a beginner more time and effort to remember forgotten code than it will to ignore unnecessary comments.


\end{enumerate}

\end{document}