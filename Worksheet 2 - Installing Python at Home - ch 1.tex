%%%%%%%%%%%%%%%%%%%%%%%%%%%%%%%%%%%%%%%%%
% Short Sectioned Assignment
% LaTeX Template
% Version 1.0 (5/5/12)
%
% This template has been downloaded from:
% http://www.LaTeXTemplates.com
%
% Original author:
% Frits Wenneker (http://www.howtotex.com)
%
% License:
% CC BY-NC-SA 3.0 (http://creativecommons.org/licenses/by-nc-sa/3.0/)
%
%%%%%%%%%%%%%%%%%%%%%%%%%%%%%%%%%%%%%%%%%

%----------------------------------------------------------------------------------------
%	PACKAGES AND OTHER DOCUMENT CONFIGURATIONS
%----------------------------------------------------------------------------------------

\documentclass[paper=a4, fontsize=11pt]{scrartcl} % A4 paper and 11pt font size

\usepackage[T1]{fontenc} % Use 8-bit encoding that has 256 glyphs
\usepackage{fourier} % Use the Adobe Utopia font for the document - comment this line to return to the LaTeX default
\usepackage[english]{babel} % English language/hyphenation
\usepackage{amsmath,amsfonts,amsthm} % Math packages
\usepackage{titling}
\usepackage{lipsum} % Used for inserting dummy 'Lorem ipsum' text into the template

\usepackage{multicol}
\setlength{\columnsep}{1cm}

\usepackage{enumitem}

\usepackage{framed}
\usepackage{listings}

\usepackage{sectsty} % Allows customizing section commands
\allsectionsfont{\centering \normalfont\scshape} % Make all sections centered, the default font and small caps

\usepackage{fancyhdr} % Custom headers and footers
\pagestyle{fancyplain} % Makes all pages in the document conform to the custom headers and footers
\fancyhead{} % No page header - if you want one, create it in the same way as the footers below
\fancyfoot[L]{} % Empty left footer
\fancyfoot[C]{} % Empty center footer
\fancyfoot[R]{\thepage} % Page numbering for right footer
\renewcommand{\headrulewidth}{0pt} % Remove header underlines
\renewcommand{\footrulewidth}{0pt} % Remove footer underlines
\setlength{\headheight}{13.6pt} % Customize the height of the header

\numberwithin{equation}{section} % Number equations within sections (i.e. 1.1, 1.2, 2.1, 2.2 instead of 1, 2, 3, 4)
\numberwithin{figure}{section} % Number figures within sections (i.e. 1.1, 1.2, 2.1, 2.2 instead of 1, 2, 3, 4)
\numberwithin{table}{section} % Number tables within sections (i.e. 1.1, 1.2, 2.1, 2.2 instead of 1, 2, 3, 4)

\setlength\parindent{0pt} % Removes all indentation from paragraphs - comment this line for an assignment with lots of text

%----------------------------------------------------------------------------------------
%	TITLE SECTION
%----------------------------------------------------------------------------------------

\newcommand{\horrule}[1]{\rule{\linewidth}{#1}} % Create horizontal rule command with 1 argument of height

\title{	
\normalfont \normalsize 
\textsc{Northridge Preparatory School} \\ [25pt] % Your university, school and/or department name(s)
\horrule{0.5pt} \\[0.4cm] % Thin top horizontal rule
\huge Worksheet\#2 : Python 3 at home \\ % The assignment title
\horrule{2pt} \\[0.5cm] % Thick bottom horizontal rule
}

\preauthor{}
\postauthor{}
\author{} % Your name

\predate{}
\postdate{}
\date{} % Today's date or a custom date

\begin{document}

\maketitle % Print the title

%----------------------------------------------------------------------------------------
%	PROBLEM 1
%----------------------------------------------------------------------------------------

\section{Installing Python at Home}

If you have Windows XP or earlier, or a non-Windows operating system, talk to Mr. Rodriguez.  Otherwise, follow these directions to install Python 3 on your home computer:

\begin{enumerate}


\item Go to: https://www.python.org/downloads/

\item Hover over the "Downloads" tab, and under "Download for Windows" click on Python 3.5.1

\item Find and open the downloaded file: python-3.5.1.exe

\item Follow the on-screen directions for a regular installation


\end{enumerate}

You should now have Python 3 installed on your computer.  If you can't find it, search for "IDLE", and that should bring up the Python interface.  It will probably be useful to right-click on the search result and add a shortcut to your desktop.\\


%[More info here?]



\section{Finding your saved documents folder}

It can sometimes be a challenge to locate the saved files you have created with Python.  Follow these directions to establish an easy-to-find folder:

\begin{enumerate}

\item Open Python, and in the File menu choose New File.  This opens a new script.

\item In the new script, under the File menu choose "Save As...".  A Save window will pop up.

\item Just under the window title "Save As" is the folder address bar.  It should end with the name Python35-32.  Right-click on this folder name, and select "Copy Address".

\item Right-click on an empty space on your desktop, and choose "Paste Shortcut".

Now you can quickly access your saved file folder.

\end{enumerate}


%[Practice moving, renaming, exporting docs ?]


\pagebreak

\section{Testing Out Your Shell}

Do the following steps to practice debugging with Python on your own computer.

\subsection{Importing and Running a Script}



\begin{enumerate}

\item Check your email for a message entitled "Worksheet 2 - Faulty Script", and download the attachment entitled "Worksheet 2 - Faulty Script.py".  This is a script of code with errors; it will not run properly.  

\item The code below is a correct version of the script you just downloaded.  Use it to proofread your faulty script, and correct the 8 errors it contains.  

\item Run your debugged script and record the 3x3 matrix output here:

\end{enumerate}



\vspace{1cm}


\begin{framed}
\begin{lstlisting}[language=Python]
# Code Start

# 3x3 matrix
X = [[12,7,3],
    [4 ,5,6],
    [7 ,8,9]]
# 3x4 matrix
Y = [[5,8,1,2],
    [6,7,3,0],
    [4,5,9,1]]
# result is 3x4
result = [[0,0,0,0],
         [0,0,0,0],
         [0,0,0,0]]

# iterate through rows of X
for i in range(len(X)):
   # iterate through columns of Y
   for j in range(len(Y[0])):
       # iterate through rows of Y
       for k in range(len(Y)):
           result[i][j] += X[i][k] * Y[k][j]

for r in result:
   print(r)

\end{lstlisting}
\end{framed}



\end{document}